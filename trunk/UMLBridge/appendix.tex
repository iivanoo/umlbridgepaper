
\section{Appendix}

The demonstration will be carried out using two projectors to
provide both a technical and a practical perspectives in parallel.
In the following, a description of the demonstration is given steps
by step.

\subsection{\em Software Architectures and Architecture Description Languages}
We will start the demonstration by giving a short introduction to
software architectures and the existent languages and tools to
describe software architectures. A table summarizing existing
approaches will be presented with the aim of providing a snapshot of
the state of the art and of the practice in this area.
In this step ADLs are divided into two main types: first-generation and second-generation ones.
The requirements a next-generation ADL should satisfy are also provided.

\subsection{\em The role of MDE technologies into next-generation of ADLs}
We will explain the power and the potentiality of Model-driven
technologies and how they can open new perspective in software
architectures and software engineering in general.

\subsection{\em Technologies overview}
 As the audience might not be
familiar with model transformation techniques and with the AMMA
platform, their characteristics will be introduced and shown in
details. The Eugenia editor creator is presented, along with  a brief sketch on
the HUTN textual notation.

\subsection{\em The \name{} framework}
The parts that compose the \name{} framework will be conceptually
shown on projector 1 and practically shown on projector 2 (see
Figure~\ref{fig:highLevelDesign}). By means of projector 2 we will
illustrate how the various parts of the \name{} framework have been implemented
and how they relate to each other within the Eclipse platform.

\subsection{\em Putting in practice \name{}}
In this presentation step we show how the conceptual features of
\name{} are applied to an illustrative case study, explaining the
usage session of our framework both for creating a next generation ADL
and for using it to model a software system.

\subsubsection{\em Composing the notations}
In this part of the demonstration we will compose (see Section {\bf Metamodels composition:}) the Darwin ADL\footnote{J. Magee and
J. Kramer. Dynamic structure in software architectures. SIGSOFT
Softw. Eng. Notes, 21(6):3 14, 1996.}
with other metamodels in order to enrich it with additional concepts.
In this step we will show how metamodels are imported into the \name{} framework
(see the {\bf Metamodels import} section in the paper).

The first step is to consider the Darwin ADL as the staging point for the
composition. Now we enrich it with additional concerns
like fault tolerance, direct link to the development process, and so
on. We do this incrementally, by firstly adding to Darwin the
idealized fault tolerant component model\footnote{G. Ferreira, C. M.
Rubira, and R. de Lemos. Explicit Representation of Exception
Handling in the Development of Dependable Component-based Systems.
In HASE01, 2001.} that is one of the solutions extensively used to
make an architecture fault tolerant. Then we integrate Darwin
extended with fault tolerance with the development process in
Business Process Modeling Notation (BPMN)\footnote{BPMN
specification: \small{\url{http://www.bpmn.org/}}} and finally the
obtained ADL is customized by adding software connectors as
first-class elements.

% IdealComponent
\noindent \emph{DarwinFT: Extending Darwin with Fault Tolerance}. We
compose the Darwin ADL with the \textit{IdealComponent}
model\footnote{D. Di Ruscio, H. Muccini, A. Pierantonio, and P.
Pelliccione. Towards weaving software architecture models. In
MBD-MOMPES '06, 2006.} for specifying software components according
to the idealized fault tolerant component model.
Figure~\ref{fig:WM_DarwinIdealComponent} is a screenshot of the
weaving model composing the Darwin and \textit{IdealComponent}
metamodels.

% BPMN
\noindent \emph{DarwinFT+BPMN: DarwinFT \& Development process in}
\emph{BPMN}. In this scenario, we show how DarwinFT can be composed
with the BPMN metamodel. Upon doing so, software architects can
associate structural parts of the system to specific development
activities. In this step we use the Eclipse STP BPMN simplified metamodel\footnote{STP BPMN simplified metamodel: \underline{http://www.eclipse.org/bpmn}} and
Figure~\ref{fig:WM_DarwinFTBPMN} is a screenshot of the weaving model
composing the DarwinFT and BPMN metamodels.

% customization
\noindent \emph{(DarwinFT+BPMN)$_{cc}$: Darwin customization}. We
customize the DarwinFT+BPMN language by adding software connectors
as first-class elements. Components may communicate also through
connectors now and connectors have associated portals that act as
architectural roles.
Figure~\ref{fig:WM_DarwinBPMNCustomization} is a screenshot of the weaving model
customizing the DarwinFT+BPMN metamodel by adding the concept of software connector.

\subsubsection{\em Generating the model migrators}
In this step we will consider the weaving model of the first scenario
(i.e., the one composing the Darwin metamodel with the \textit{IdealComponent} UML profile)
and we will generate a model migrator from it.
More specifically, we will generate the model migrator that produces a Darwin and UML model starting from
a model conforming to the composed metamodel (the one we call DarwinFT).
Figure~\ref{fig:migratorScreenshot} shows part of the generated migrator in our Eclipse
environment.
Depending on the time availability, we will consider an example model conforming to
the composed metamodel and we will execute the generated migrator
and a generic Darwin specification extractor on that model
in order to show the audience that a ready-to-use Darwin specification is produced by the migrator.

\subsubsection{\em Generating the editors}
In this section we will present the various concrete syntaxes of the language
we are developing. We will show the three editors that the current \name{} framework
is able to provide for the generated ADL; they are the tree-based (see Figure~\ref{fig:treeEditor}),
textual (see Figure~\ref{fig:textualEditor}) and graphical one (see Figures~\ref{fig:graphicalEditor1}~\ref{fig:graphicalEditor2}), respectively.

\subsubsection{\em Using the generated language}
In this part of the demonstration we will model a software system
by means of the new ADL we have constructed (DarwinFT+BPMN$_{cc}$)
using the editors generated in the previous step.
The modeled system is called Integrated Environment for Communication on Ship
(IECS)~\cite{duallyTSE}; it is based on a specification that comes from
a project developed within Selex Communications, a company mainly
operating in the naval communication domain.

IECS is a multitier environment capable of maintaining a fail-safe
communication within a military vessel. The main functionalities of
the system are: (i) provide several communication modes; (ii) manage
and distribute operative messages; (iii) manage and configure
transmission and reception over radio channel; (iv) remote control
and monitoring of the system; (v) implement communication security
techniques.

Figures~\ref{fig:treeEditor},~\ref{fig:textualEditor},~\ref{fig:graphicalEditor1},~\ref{fig:graphicalEditor2}
give an idea on the software architecture of the IECS system.
The integration of Darwin with BPMN, allows us to model both
%(see Figure \ref{fig:editors}.c)
the software architecture of the IECS system and the adopted
development strategies. The IECS software architecture is composed
of the Equipment, Workstation, Proxy, DB and CTSM components. The
type of the latter one (CTSMType in Figure~\ref{fig:graphicalEditor1}) is described using the
idealized fault tolerant component model integrated in Darwin.
Network is a software connector; this is possible thanks to the
customization of Darwin we performed (third composition scenario).
%As can beseen in figure, Darwin and BPMN are represented via two different
%views. (DarwinFT+BPMN)$_{cc}$ enables to assign software components
%of the designed SA to developer teams. In particular, let us assume
%to have two developers teams (namely, A and B), a system engineering
%team and a testing team. The dotted lines graphically render how
%Darwin components are associated to the BPMN tasks and pools
%describing the activities assigned to each team (i.e., Proxy, DB,
%and Network are assigned to the system engineering team, Equipment
%and CTSM to the development team A, and Workstation to the
%development team B). Even if these relationships are not graphically
%rendered, they exist in the IECS model and can be accessed through
%the Properties panel of both graphical editors.

\subsection{\em Relationship with the next-generation requirements}
In this step we aim to show to the audience how the newly generated ADL
relates to the requirements a next-generation ADL should satisfy.
We will analyse each requirement independently and we will show how it relates to the generated ADL
by means of practical examples.


\subsection{\em Future work}
In this step we will present the future work we envision on the \name{} framework.
We decided to present the future work on \name{} in this demonstration in order
to discuss it with the audience about it and possibly to gain useful ideas to
enhance the framework.

The future work on \name{} comprises the following directions:
\begin{itemize}
    \item We are planning to investigate more powerful and formal ways to provide semantics.
%    In the current version of \name{}, the semantics of the ADL is defined by means
%    of relationships with $A_0$.
    Indeed, since $A_0$ is minimal, there could be some problems related
    to its expressivity (e.g., the semantics of elements that do not have a corresponding one in $A_0$ is
    not precise).
    \item
    %There is no evidence that the defined operators are enough for extending any existing ADL.
    We are investigating on the completeness of the
    set of composition operators and on how to prove that a candidate operators set is complete.
    \item The graphical editors creator is the most prototypical component of \name{},
    we will investigate on how to improve it.
%    It is not able to automatically solve every possible conflict that may arise. This
%    aspect needs further investigation.
    \item \name{} allows the creation of new ADLs by extending existing ones. We will investigate on the creation of a
    new ADL from scratch (possibly starting from the $A_0$ metamodel), and we will assess its feasibility as well.
%    \item After the execution of a migrator, a model conforming to the native ADL is generated.
%    We will work on how to provide means to identify the parts of the software architecture described using the
%    generated ADL that are affected by changes made by native ADL tools.
%    This enables software architects to understand if results of performed
%    analysis are still valid even after changes or if the analysis must be re-performed.
\end{itemize}

\subsection{\em References}
Finally, the web site of \name{} will be shown to allow the
audience to know where they can download the framework and the case
studies as well as to find more details about the tool that is under
construction and the release dates.


Depending on the time availability, some of the aforementioned items
may be shortened or deleted.

%\newpage

\section{Screen Dumps}\label{sec:dumps}
%(see next pages)
%
%\begin{figure}[h]
%    \begin{center}
%    \includegraphics[scale=.45]{figures/Darwin.png}
%    \caption{Darwin metamodel}\label{fig:Darwin}
%    \end{center}
%\end{figure}
%
%\begin{figure}[h]
%    \begin{center}
%    \includegraphics[scale=.45]{figures/IdealComponent.png}
%    \caption{Ideal Component UML profile}\label{fig:customization}
%    \end{center}
%\end{figure}
\begin{figure*}[ht]
    \begin{center}
    \includegraphics[scale=.3]{figures/WM_DarwinIdealComponent.png}
    \caption{Composing the Darwin metamodel with the \textit{IdealComponent} UML profile.}\label{fig:WM_DarwinIdealComponent}
    \end{center}
\end{figure*}
%\begin{figure*}[h]
%\begin{center}
%    \includegraphics[scale=.4]{figures/BPMN.png}
%  \caption{BPMN metamodel}\label{fig:BPMN}
%\end{center}
%\end{figure*}
%
%\begin{figure}[h]
%    \begin{center}
%%    \includegraphics[scale=.45]{figures/importWizard.png}
%    \caption{Import metamodels wizard}\label{fig:importWizard}
%    \end{center}
%\end{figure}

\begin{figure*}[ht]
    \begin{center}
    \includegraphics[scale=.3]{figures/WM_DarwinFTBPMN.png}
    \caption{Composing the DarwinFT metamodel with the BPMN metamodel.}\label{fig:WM_DarwinFTBPMN}
    \end{center}
\end{figure*}

%\begin{figure*}[h]
%\begin{center}
%    \includegraphics[scale=.4]{figures/softwareConnectorMM.png}
%  \caption{Software connector metamodel}\label{fig:softwareConnectorMM}
%\end{center}
%\end{figure*}

\begin{figure*}[ht]
    \begin{center}
    \includegraphics[scale=.3]{figures/WM_DarwinBPMNCustomization.png}
    \caption{Customization of the DarwinFT+BPMN metamodel by adding software connectors.}\label{fig:WM_DarwinBPMNCustomization}
    \end{center}
\end{figure*}

\begin{figure*}[ht]
    \begin{center}
    \includegraphics[scale=.3]{figures/migratorScreenshot.png}
    \caption{The \textit{Composed2Single} migrator generated by \name{} from the \textit{Darwin\_IdealComponent} weaving model.}\label{fig:migratorScreenshot}
    \end{center}
\end{figure*}

\begin{figure*}[ht]
    \begin{center}
    \includegraphics[scale=.3]{figures/treeEditor.png}
    \caption{The tree-based editor for the generated ADL.}\label{fig:treeEditor}
    \end{center}
\end{figure*}

\begin{figure*}[ht]
    \begin{center}
    \includegraphics[scale=.3]{figures/textualEditor.png}
    \caption{The textual editor for the generated ADL.}\label{fig:textualEditor}
    \end{center}
\end{figure*}

\begin{figure*}[ht]
    \begin{center}
    \includegraphics[scale=.3]{figures/graphicalEditor1.png}
    \caption{The structural part of the graphical editor for the generated ADL.}\label{fig:graphicalEditor1}
    \end{center}
\end{figure*}
\begin{figure*}[ht]
    \begin{center}
    \includegraphics[scale=.3]{figures/graphicalEditor2.png}
    \caption{The BPMN-related part of the graphical editor for the generated ADL.}\label{fig:graphicalEditor2}
    \end{center}
\end{figure*}
