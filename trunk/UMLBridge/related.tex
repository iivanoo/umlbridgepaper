\section{Related work}\label{sec:related}
%Both UML profiling and DSML-based approaches are broadly used in practice, and the need to reconcile these two
%modeling techniques is well recognized in research. 
Many approaches to bridge UML profiles and MOF metamodels have been proposed in research, each of them with its own features and usage scenarios. 
%They mainly differ from our bridge in terms of level of automation, and of the effort required by designers to use them.

Abouzahra et al.~\cite{Abouzahra} proposed an integration process starting from a UML profile, a MOF metamodel and a weaving model linking their concepts.
%Their concepts (e.g., stereotypes and tagged values for UML profiles, and metaclasses and structural features for MOF metamodels)
%are linked by means of a weaving model. 
%Such a model is also produced by the user and specifies a set of links between elements
%of the UML profile and elements of the MOF metamodel. 
The tool takes as input the three artifacts and generates two ATL transformations that enable the bridge to be executed at the modeling level. Similarly to our work, the ATL transformations are generated by means of higher-order transformations. Further, also their approach works both at the metamodeling and modeling level, and it is based on the AMMA platform. In spite of these commonalities, the two approaches are conceptually different. While in our approach a UML profile is the only input required, the approach of Abouzahra et al. requires designers to define the MOF metamodel and the weaving model linking it to the initial UML profile too.
Moreover, while in our approach the links between UML profiles and MOF metamodels are defined at the MOF level, independently from the UML profile to be bridged, in Abouzahra et al. correspondences between each UML profile and its corresponding metamodel must be manually defined; that is, there is no normalized way to define the mappings between UML profiles and MOF metamodels.

In~\cite{Wimmer}, a specular approach is proposed: the authors assume to start from the DSML metamodel, and then to automatically
generate its corresponding UML profile; model transformations for transforming DSML models to UML models and vice versa are also generated. Similarly to Abouzahra's work, this approach is based on the manual definition of a weaving model between the profile and the DSML metamodel.
%which maps each element of the DSML metamodel to the corresponding element in the UML metamodel. 
%Again, our approach differs from this related work since the mapping between the DSML metamodel and the UML profile is defined at the MOF level, thus enabling us to propose a fully automatic mechanism. 
Moreover, the two approaches proposed in~\cite{Abouzahra} and~\cite{Wimmer} have a common limitation: since the weaving model is defined by the user, elements for which there is no mapping in the weaving model are lost while round-tripping from UML to the DSML metamodel. Basically, our approach does not suffer from this limitation since the MOF metamodel is automatically generated and the mappings are defined at the M3 level. The only case in which our approach may experience the lost of information problem is when the slicing mechanism is used with wrong assumptions made by designers. 
%For example, let us suppose that a designer assumes that only UML component diagrams are of interest, and so elements pertaining to other UML diagrams are sliced out from the generated MOF metamodel. In case a state machine is defined for each component, when transforming UML to the corresponding MOF metamodel, information about state machines is lost (since sliced out). Clearly, in this scenario the lost of information problem is unavoidable, and related to the designer wrong assumption.

In~\cite{Graaf}, UML is used as a notation to visualize models conforming to a given MOF-based DSL.
More specifically, this approach presents a unidirectional mapping from models conforming to the DSL to profiled UML models.
By means of this technique it is possible to visualize DSL models into standard UML editors;
this transformation is performed for documentation purposes.
 %because designers are more accustomed to the UML concrete syntax.
This approach differs from ours in many points: (i) the whole process starts from the DSL metamodel, while the initial artifact of our bridge is the definition of the UML profile, (ii) its goal is to support the documentation of DSL models, while our bridge aims at supporting the manipulation of UML models, and (iii) it proposes a unidirectional transformation from MOF to UML, while the transformations of our bridge operate in both directions (i.e., from UML to MOF and vice versa).

In the Eclipse platform a transformation from UML profiles to MOF metamodels is provided by EMF. A Java class called \texttt{Profile2EPackageConverter} implements such a mechanism and  can be executed via a standard Java method call.
This class converts UML profiles to representative Ecore packages. The transformation logic is similar to the one we proposed at the metamodeling level (in Section~\ref{sec:metamodelLevel}), but it does not provide any mechanism to execute the bridge at the modeling level. This implies that designers can use \textit{Profile2EPackageConverter} to automatically convert a UML profile into an Ecore metamodel, but they are forced to consider each UML model and to manually rebuild it as a model conforming to the newly created Ecore metamodel. 