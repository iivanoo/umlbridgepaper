\section{Related work}\label{sec:related}

Both UML profiling and Domain DSML-based approaches are broadly used in practice, and the need to reconcile these two
modeling techniques is well recognized in research. 
Many approaches to bridge UML profiles and MOF metamodels have been proposed, each of them with its own specific features and 
usage scenarios. 
They mainly differ from our bridge in terms of level of automation, and of the effort required by designers to use them.

Abouzahra et al. \cite{Abouzahra} proposed an integration process starting from a UML profile and a MOF metamodel.
Their concepts (e.g., stereotypes and tagged values for UML profiles, and metaclasses and structural features for MOF metamodels) 
are linked by means of a weaving model. Such a model is also produced by the user and specifies a set of links between elements
of the UML profile and elements of the MOF metamodel.
The tool takes as input then UML profile, the MOF metamodel and the weaving model linking them and it generates two ATL transformations that allow
the bridge to be executed at the modeling level. Similarly to our work, the ATL transformations are generated by means of 
higher-order transformations. Another commonality with our work is that their approach works both at the metamodeling and modeling level,
and that both are implemented in the context of the AMMA platform.
In spite of these commonalities, the two approaches differ a lot conceptually. More specifically,
since the starting point of our bridge is the UML profile only, our approach is completely automatic;
on the contrary,  the approach of Abouzahra et al. requires designers to define
also the MOF metamodel and the weaving model linking it to the initial UML profile.
Moreover, in our approach the links between UML profiles and MOF metamodels are defined at the MOF level (i.e., M3), and so
they are defined once and for all, independently from the UML profile to be bridged; differently, 
in the approach of Abouzahra et al. correspondences between each UML profile and its corresponding metamodel must be manually defined; 
that is, there is no normalized way to define the mappings between UML profiles and MOF metamodels.

In \cite{Wimmer}, a specular approach is proposed: authors assume to start from the DSML metamodel, and then to automatically
generate its corresponding UML profile; model transformations for transforming DSML models
to UML models and vice versa are also generated. Similarly to Abouzahra's work, this approach is based on the manual definition of a weaving model which maps
each element of the DSML metamodel to the corresponding element in the UML metamodel.
Again, our approach is different since the mapping between the DSML metamodel and the UML profile is defined at the MOF level; this
allows us to propose a fully automatic mechanism. Also, the two approaches proposed in \cite{Abouzahra} and \cite{Wimmer} share a common
limitation to what regards the lost information during round-trip: since the weaving model is defined by the user, 
elements for which there is no mapping in the weaving model are lost while transforming back and forth from UML to the DSML metamodel. 
Basically, our approach does not suffer from this limitation since the DSML metamodel is automatically generated and the mappings are defined at the M3 level. The only case in which our approach may experience the lost information problem is when the slicing mechanism is used and the initial assumptions made by designers do not hold on the models. For example, let us suppose that designers assume that only UML component diagrams
are used, so elements pertaining to other UML diagrams are sliced out from the generated MOF metamodel; however, it may happen that in some model a state machine is defined into each component; when applying the transformation from UML to the MOF metamodel, all the information
related to state machines is lost. Clearly, in this kind of situation the lost information problem is unavoidable, because it is a consequence of the wrong assumption of the designer.

In \cite{Graaf}, UML is used as a notation to visualize models conforming to a given MOF-based DSL.
More specifically, this approach presents a unidirectional mapping from models conforming to the DSL to profiled UML models.
By means of this technique it is possible to visualize DSL models into standard UML editors;
this transformation is performed for documentation purposes because designers are more accustomed to the UML concrete syntax.
This approach differs from ours in many points: 
(i) the whole process starts from the DSL metamodel, while the initial artifact of our bridge is the definition of the UML profile,
(ii) its goal is to support the documentation of DSL models, while our bridge aims at supporting the manipulation of UML models, and
(iii) it proposes a unidirectional transformation from MOF to UML, while the transformations of our bridge operate in both directions
(i.e., from UML to MOF and vice versa).

In conclusion, in the context of the Eclipse platform an automatic mechanism for bridging UML profiles and MOF metamodels is provided by EMF.
It is implemented as a Java class called \texttt{Profile2EPackageConverter} and  can be executed via a standard Java method call. 
This class converts UML profiles to representative Ecore packages; the transformation logic is similar to the one we proposed at the metamodeling
level (see Section \ref{sec:metamodelLevel}), but it does not have any mechanism to execute the bridge at the modeling level. This implies that
designers can use \textit{Profile2EPackageConverter} to automatically convert a UML profile into an Ecore metamodel, but they are forced to consider each UML model and to manually rebuild it as a model conforming to the newly created Ecore metamodel.