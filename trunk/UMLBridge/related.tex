\section{Related work}\label{sec:related}
\ivano{descrizione di lavori simili in letteratura e come si differenziano da noi}

The problem of interoperability has already been addresses in previous researches. Abouzahra, B\`ezivin et al. \cite{Abouzahra}, propose an integration process starting with a UML profile. 

The tool, they implemented, takes as input an UML profile and the metamodel of the system described by the profile. That tool allows
transforming between models conforming to those inputs. It transforms UML models designed with this profile to models conforming to the profiled metamodel and vice versa. The tool aims at automatically generating transformations between models and not transforming a metamodel to a profile or the opposite. To reach this goal it needs mapping links to be implemented. Mapping details specify links between UML elements and the elements from the profiled metamodel. This information should not to be redundant. To meet this requirement, this approach represents the two input models (the profile model and the profiled metamodel) separately. A modeling weaving tool, the AMMA model weaving one, has been used for this purpose. This approach has the limitation of finding correspondaces beetwen the profile and the metamodel, as mentioned in the limitation section of their paper, since there's no normalized way to define UML profiles and their mapping with MOF metamodels. What we propose is a tool wich bridges, in a completely automated manner, a profile to metamodels (and every profiled model into the corresponding). This previous approach led us the way to our approach.


Another approach has been proposed by Manuel Wimmer in this \cite{Wimmer}. Assuming there's no available profile yet, this approach proposes a Bridge Generator component which automatically generates transformations and profile by means of an explicit, and manually build, mapping model. The integration process then is as follows: First, the user defines the correspondences between the DSL metamodel and the UML metamodel in terms of a mapping model in an interactive mapping environment. The mapping model is expressed with a dedicated metamodel bridging language which enables the automatic processing. After finalizing the mapping model, the Bridge Generator automatically generates the UML profile and in the case that an uni-directional model transformation language is used, also the transformations from the DSL to UML and back again. Wimmer's approach has, as its principal benefit, the reduction of error having a single source of information (the mapping model) and automated repetitive tasks by means of the transformation. Our approach, again, has the completely automated manner of defining the mapping. Moreover, our approach is in a way specular to Wimmer's one, since our starting point is a profile and several profiled models.


Another method used for bridging purposes is the use of EMF umluitl.java library. This library converts UML elements to representative Ecore model elements. This utility helps, in the Eclipse framework, to automatically convert UML elements into Ecore ones. Nevertheless the conversion is possible only at the metamodeling level, one developer has to manually implement transformations from his own profile to the metamodel that is the strenght of the bridge we are propose here.

\marco{reference al sito? http://download.eclipse.org/modeling/mdt/uml2/javadoc/2.2.0/org/eclipse/uml2/uml/util/UMLUtil.UML2EcoreConverter.html}