\section{The automatic bridge}\label{sec:framework}
%
%The proposed bridge allows to automatically consider a UML profile as standard MOF metamodel.
%So, on one side software designers to develop models using UML profiles familiar to them, and on the other side
%developers of tools that manipulate models can assume to work on MOF metamodels only, avoiding the additional
%constraints introduced by UML profiling.

Figure~\ref{fig:overall} provides a high-level view on how the proposed bridge works. The starting point of the whole bridging mechanism is a UML profile and models conforming to it (see Figure~\ref{fig:overall}). The profile and its models can be developed using standard UML modeling tools. Then, all the other modeling artifacts involved in the bridge are automatically generated, and specifically:
\begin{enumerate}
	\item the MOF $MM$ metamodel containing all the concepts corresponding to the elements of the UML profile,
	\item a set of model-to-model transformations enabling the transformation of profiled UML models into models
conforming to the MOF metamodel, and vice versa.
\end{enumerate}


\begin{figure}[htbp]
	\centering
		\includegraphics[width=0.85\textwidth]{figures/overview.png}
	\caption{High-level view of the proposed bridge}
	\label{fig:overall}
\end{figure}


Modeling artifacts involved in the bridging procedure lie at different levels of abstraction
(i.e., metamodeling and modeling, in the MDE modeling stack). In order to keep the bridging procedure well defined and manageable,
we decided to keep this distinction in our bridge by decomposing it into two main phases:
%
\begin{itemize}
	\item[$\bullet$] \textbf{phase 1}: it is performed at the metamodeling level, and consists in the generation of $MM$ from the UML profile.
	\item[$\bullet$] \textbf{phase 2}: it works at the modeling level, and consists in the generation of model-to-model
	transformations between UML and $MM$.
\end{itemize}

Next sections will describe in details the two phases of the bridging procedure.
It is important to note that the proposed bridge is \textit{completely automatic}. That is, when a UML profile is defined, the above mentioned bridging phases do not require any additional effort to the user. As a matter of fact, the designer works on the UML side, while tool developers work on standard MOF metamodels, thus enabling a clear separation of concerns.


\subsection{Phase 1: The bridge at the metamodeling level}\label{sec:metamodelLevel}

At the metamodeling level, the bridge takes an initial UML profile and generates {\em a} corresponding MOF metamodel.
As shown in Figure~\ref{fig:metamodelingLevel}, a model-to-model transformation called
\textit{UMLprofile2MOF} is used for this purpose. This transformation generates a MOF metamodel ($MM_x$ in figure) starting from (i) the definition of the UML profile and (ii) the UML metamodel. The latter input is needed since the transformation has to access UML metaclasses referenced by the various stereotypes of the profile.

\begin{figure}[htbp]
	\centering
		\includegraphics[width=0.60\textwidth]{figures/metamodelingLevel.png}
	\caption{The bridge at the metamodeling level}
	\label{fig:metamodelingLevel}
\end{figure}


%
The \textit{UMLprofile2MOF} transformation contains a transformation rule for each element that can appear in the definition of a UML profile. In the following we describe how the main transformation rules behave:

\begin{itemize}
	\item[$\bullet$] \textbf{Stereotype2Class}. Each UML stereotype is mapped into a MOF metaclass
	(e.g., \textit{XComponent} in Figure \ref{fig:metamodelingExample}).  		
	Tagged values of the source stereotype are separately managed by the \textit{Property2Feature} rule (described below).
	The rule also checks whether the source UML stereotype specializes some other elements, and recreates the generalization hierarchies in the target MOF metamodel, accordingly.
	A special reasoning is applied to transform the relationship between a stereotype and the UML metaclasses it extends.
	According to the UML superstructure~\cite{UML}, \textit{"the MOF construct equivalent to an extension is an aggregation from
	the extended metaclass to the extension stereotype"}. So, this rule transforms each UML extension into
	a MOF containment reference with cardinality \textit{0..1}. 	
	\item[$\bullet$] \textbf{Class2Class}, each UML class is mapped to a MOF metaclass.
	Properties of the source UML class are managed by the \textit{Property2Feature} rule and, similarly to \textit{Stereotype2Class},
	contains a specific mechanism for managing its generalization hierarchy.
	\item[$\bullet$] \textbf{Profile2Package}, each UML profile is mapped to a MOF package (e.g.,the \textit{ XProfile} in figure). Specific bindings populate the newly generated package with the correct metaclasses (generated either by the \textit{Class2Class} or the \textit{Stereotype2Class} rules) in order to recreate the same hierarchy of containments.
	\item[$\bullet$] \textbf{Package2Package}, each UML package is mapped to a MOF package. It is populated in a similar way as in the \textit{Profile2package} rule.
	\item[$\bullet$] \textbf{Property2Feature}, each UML property (like the \textit{executionTime} in figure)
	is mapped to a MOF attribute or reference depending on their type;
	more specifically, if the type of the source property is either a data type, an enumeration or a primitive type,
	then a MOF attribute is generated, otherwise the rule generates a MOF reference.
	\item[$\bullet$] \textbf{DataType2Datatype}, each UML data type is mapped to a MOF data type.
	\item[$\bullet$] \textbf{Enumeration2Enumeration}, each UML enumeration (\textit{type} in figure)
	is mapped into a MOF enumeration containing the same literals.
\end{itemize}


\begin{figure}[htbp]
	\centering
		\includegraphics[width=0.60\textwidth]{figures/metamodelingExample.png}
	\caption{Example of generated MOF metamodel}
	\label{fig:metamodelingExample}
\end{figure}


\textit{UMLprofile2MOF} also contains a set of auxiliary rules and bindings (e.g., the one setting the name of each MOF meta-element with the name of the corresponding source UML element, etc.), not described in this paper for sake of simplicity. Furthermore, \textit{UMLprofile2MOF} creates an ad-hoc package in $MM_x$ where it copies all the UML metaclasses in there. In this way, the generated MOF metamodel is self-contained, and does not depend on the UML metamodel itself.
%; this opens for the possibility to "slice" the MOF metamodel, allowing to do not suffer from the complexity of the UML metamodel any more.This aspect of the bridge is presented in Section~\ref{sec:slicing}.

%Figure \ref{fig:metamodelingExample} represents a trivial UML profile translated into a MOF metamodel. The Profile has been translated into a MOF package, the Component metaclass has been copied to the target metamodel and extended through the \textit{extension\_MyStereotype} attribute. Eventually the Stereotype has been translated into a MOF metaclass and bound with the extended metaclass by means of the composition reference.
%%


\subsection{Phase 2: The bridge at the modeling level}\label{sec:modeLevel}

At the modeling level, the proposed bridge automatically generates model-to-model transformations between UML profiles and MOF metamodels.
More specifically, Figure~\ref{fig:modelingLevel} shows the two model transformations generated by our bridge:

\begin{itemize}
	\item $UML2MM_x$ returns a model conforming to $MM_x$, starting from a UML model conforming to the $profile.uml$ profile;
	\item $MM_x2UML$ performs the opposite task: it takes as input a model conforming to $MM_x$
	and produces UML models profiled according to $profile.uml$.
\end{itemize}

\begin{figure}[htbp]
	\centering
	\includegraphics[width=0.60\textwidth]{figures/modelingLevel.png}
	\caption{The bridge at the modeling level}
	\label{fig:modelingLevel}
\end{figure}

$UML2MM_x$ has the following high-level logical structure: (i) a set of rules transform each standard UML element into an instance of its corresponding metaclass in $MM_x$; (ii) another set of rules transform each stereotype in the UML profile into an instance of the corresponding metaclass in $MM_x$; and (iii) for each rule, some imperative code is generated in order to automatically manage the application of the stereotypes. $MM_x2UML$ works in the other way round: (i) firstly it applies the UML profile to the target UML model; (ii) then, it transforms each instance of the $MM_x$ metamodel into an instance of the corresponding UML element; (iii) stereotypes are applied to  previously generated elements according to the definition of the UML profile. Specific rules manage the order in which profiles and stereotypes are applied, and how tagged values are accessed in the models. Figure~\ref{fig:modelingExample} shows an example of models produced by the bridge.

Such transformations are automatically generated by means of the execution of two Higher-Order Transformations
(i.e., transformations taking other transformations as input or producing
other transformations as output): $UML2MM_xGenerator$ and $MM_x2UMLGenerator$. It is important to note that higher-order transformations make the bridge generic (and so, totally automatic), so that it does not depend neither on source UML profiles nor on the generated MOF metamodels.

\begin{figure}[htbp]
	\centering
		\includegraphics[width=0.60\textwidth]{figures/modelingExample.png}
	\caption{Example of bridged models}
	\label{fig:modelingExample}
\end{figure}
%

%In the figure \ref{fig:modelingExample} we show the result of the execution of the generated transformation on a model which has applied a the example profile of figure \ref{fig:metamodelingExample}. As you can see the element is a \textit{Component} and has applied \textit{MyStereotype} stereotype. In the target model we have an instance of a \textit{Component} MOF class which has an instance of a \textit{MyStereotype} class. Every tagged value has been kept and rightly instantiated.


%------------------------------------------------------------------------------------------------------------------------
\section{Slicing the obtained metamodel}\label{sec:slicing}

As previously said, the default behavior of the bridge is to consider the whole UML metamodel and to keep its concepts also in the
target MOF metamodel. On one side this behavior ensures that our bridge is loss-less with respect to the information
provided in the models (it is our primary goal while designing the bridge);
however, on the other side it moves the complexity of the UML metamodel (it contains 246 classes and 583 properties)
into the target MOF metamodel.
In order to avoid this situation, we couple the bridge with a generic \textbf{slicing mechanism} that allows it to consider
only the \textit{subset of relevant UML concepts} when passing to the target MOF metamodel.

The subset of relevant UML concepts can be considered as the set of UML metaclasses that can be instantiated in the context of a specific project;
metaclasses outside the relevant set are never instantiated in any model.
A typical example is when designers assume that in the context of a specific project only UML
use case diagrams will be produced; the subset of relevant UML concepts contains only metaclasses
the can be used in sequence diagrams (e.g., actor, use case, association) and there is no need to keep modeling concepts
for state machines, sequence diagrams, and so on.
This kind of assumptions is recurrent in practice, thus we designed our approach so that the target MOF metamodel contains only
metaclasses corresponding to relevant UML concepts.
%indeed many UML modeling tools allow to restrict the kind of diagrams that can be created by designers.

The set of relevant metaclasses is an information related to the target MOF metamodel but it is not included in it (it depends on specific needs of designers).
Since the slicing mechanism is realized as a model transformation and considering that we were in a MDE context, the set of relevant
metaclasses is represented as an \textit{annotation model}
\footnote{An annotation model is a model containing auxiliary information about another model (the annotated model)\cite{MCDFthesis}} linked to the target MOF metamodel. The annotation model contains a link to each relevant metaclass coming from UML, the slicing mechanism uses those links to identify the metaclasses that will not be sliced out by the slicing procedure.

The current prototype of our approach provides three mechanisms to (semi) automatically obtain the annotation model that will drive the slicing procedure:

\begin{enumerate}
	\item the annotation model is automatically generated from a UML profile;
	\item designers specify only the types of UML diagram they will use, and the corresponding annotation model is automatically generated;
	\item designers provide a set of example UML models, and then the annotation model is automatically generated starting from them.
\end{enumerate}

The mechanisms are implemented as model-to-model transformations (see Section \ref{sec:tool} for more details);
each mechanism has a different level of automation and requires different input artifacts.
For example, if designers already know they will use only specific kinds of UML diagram, then option 2 seems to be the most convenient;
or else, if designers already defined some initial model, they may use the option 3.
Of course, if full control over the generated MOF metamodel is needed, designers can define the annotation model manually, that is they can create an annotation link for each metaclasses that can be instantiated in their UML models.

Once an annotation model is available, then the slicing mechanism can be performed;
it is based on a generic slicing algorithm for MOF metamodels~\cite{ICSEbyadl}. Let
$MM$ be a metamodel and let $SC$ be a subset of the elements in
$MM$; $slice$ is defined as follows:

\vspace{-.2cm}
$$slice(SC)=SC \cup \displaystyle\bigcup_{c \in
SC}{slice(neighbour(c))}$$
\vspace{-.2cm}

\noindent where $neighbour(c)$ is the set of all superclasses of
$c$, of all classes referred (both with association and
aggregation) by $c$, and of all types of attributes in $c$.
It is important to note that even though $slice$ is defined
as a set of classes, since each class contains also references
to other classes, the final result is a subset of the metamodel $MM$
with both classes and their relations.
%
\begin{figure}
  \centering
  \subfloat[]{\label{fig:slicerMM}\includegraphics[scale=0.4]{figures/slicerMM}}
 \hspace{10mm}
  \subfloat[]{\label{fig:slicerT}\includegraphics[scale=0.4]{figures/slicerT.png}}
  \caption{Slicing the MOF metamodel (a) and the generated transformations (b)}
  \label{fig:animals}
\end{figure}
%
As can be seen in Figure \ref{fig:slicerMM}, the slicing algorithm is realized as a model-to-model transformation called \textit{MMslicer}.
It that takes as input a MOF metamodel $MM_x$ and an annotation model $am_x$, and generates a new metamodel $MM_sliced$ containing
a subset of $MM_x$ according to the metaclasses referenced in $am_x$.

At this point we must consider a possible issue that may arise: once the target MOF metamodel has been sliced, the
previously generated model transformations of the bridge may refer to missing metaclasses in the metamodel.
This implies that our approach must provide a mechanism for adapting also the $UML2MM_x$ and $MM_x2UML$ transformations
to the newly sliced metamodel.
We are aware that in literature there are generic approaches managing the coupled evolution of metamodels and model transformations
\cite{TransEvolution}; however, since in our case we can assume that elements
can be only deleted from metamodels (we do not need to manage neither additions or updates), and since we need a fully automatic mechanism, we developed our minimalistic solution to automatically adapt model transformations to sliced metamodels.
Our solution takes inspiration from the EMFMigrate project\footnote{EMFMigrate project website: \small{\url{http://www.emfmigrate.org}}}
and it is based on an higher-order transformation; in this work we call it $Tslicer$.
%This kind of issue is well-known in the metamodel co-evolution research field \cite{CITAZIONI}, \footnote{http://www.emfmigrate.org}

Figure \ref{fig:slicerT} gives an idea of how the \textit{Tslicer} transformation works. It takes as input (i) the sliced metamodel
($MM_sliced$ in figure) and (ii) the model transformation to be adapted ($UML2MM_x$ in figure). $Tslicer$ automatically adapts the input transformation according to the meta-elements (i.e., metaclasses, attributes, references)
that were previously sliced out from $MM_sliced$. The grey elements in Figure \ref{fig:slicerT} show that $Tslicer$ can also be
applied on the transformation in the other direction (i.e., $MM_x2UML$); this ensures that using the slicing mechanism does not affect
the bridge in terms of automation and bidirectionality. More details on $Tslicer$ are provided in Section \ref{sec:tool}.

In conclusion, it is important to note that the whole slicing mechanism (both on metamodels and transformations)
acts as a post-processing activity of the artifacts generated by the bridge described in Section \ref{sec:framework}; this means that, according to project-specific needs, the proposed bridge and slicing mechanism can be used independently.
%Intuitively, it copies all the elements of the input transformation (like transformation rules, imperative statements, conditions, and so on)
%and ignores those elements that reference a meta-element  that are not in $MM_sliced$ anymore.

%removes those elements of the input model transformation (e.g., transformation rules, imperative statements, conditions)
%that reference a meta-element that is not in $MM_sliced$ anymore (i.e., such an element has been sliced out by $Mslicer$).
