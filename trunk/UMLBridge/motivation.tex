\section{Motivation}\label{sec:motivation}

%UML profiling and MOF-based Domain Specific Modeling Languages are the most used approaches for specifying and documenting a system. 
%Even though these two modeling approaches share many common aspects, each of them has its own set of peculiar features. 
%Next section provides some background information on them and will serve as a dictionary of terms used
%throughout the paper. In Section \ref{sec:motivation2} we discuss the main motivations that lead us to propose this work.
%
%
%\subsection{UML profiles and Domain Specific Modeling Languages}\label{sec:background}
%\ivano{descrizione di cosa e' un profilo UML e come sono utilizzati}
%
%\ivano{descrizione di cosa e' un DSML e come sono utilizzati}

As previously said, the primary motivation for proposing our bridge is to alleviate the accidental complexity of manipulating profiled UML models, 
without forcing designers to do not use UML for their modeling activities.
In this section we provide a discussion on the main motivations that lead us to propose this work.
For the sake of clarity, we categorize our observations in three high-level scenarios.

\textbf{First scenario - Analysis tools based on UML profiles}
Both in academia and in practice there are many tools that allow designers to perform some kind of analysis on profiled UML models.
For example, the XX allows to, and YYY performs...
These tools share a strong difficulty in manipulating models conforming to UML (vedere metaPruning).
The proposed bridge helps tool developers since they can reason on standard MOF metamodels and the involved tools
operate on smaller MOF-based models only, ...

\textbf{Second scenario - Model transformation tools}
Current model transformation tools do not natively support the management of UML profiles.
Thus, in current state-of-the-practice model transformation engines 
like ATL\footnote{ATL project website: \small{\url{http://www.eclipse.org/atl}}} and
MediniQVT\footnote{Medini QVT project website: \small{\url{http://projects.ikv.de/qvt}}}
need to be tailored or extended to support the transformation of profiled UML models; 
even worse, some interesting model transformation engine (e.g.,   
JTL\footnote{JTL project website: \small{\url{http://jtl.di.univaq.it}}}, GReAT) 
has the limitation that it cannot natively access profile-specific information when transforming UML models.
These problems are caused by a set of additional constraints that make the manipulation of profiled UML models difficult for both model transformations users and transformation engines developers:
%
\begin{itemize}
	\item specific mechanisms to apply (and un-apply) either UML profiles or stereotypes must be implemented 
	in the model transformation engine itself (or at least as an ad-hoc extension for it);
	\item the model transformation engine must consider the order in which profiles and stereotypes can be applied; 
	for example, a UML restriction imposes that a stereotype cannot be applied to a model element $x$ before 
	the profile has been applied to a package containing $x$;
	\item the model transformation language must expose specific constructs for accessing tagged values associated to a UML model element;
\end{itemize}
%
By automating the transition from profiled UML models to MOF-based models and vice versa, our bridge relaxes the above mentioned constraints
by allowing model transformation users and transformation engine developers to assume they work on MOF-based models only.
  
\textbf{Third scenario - Homogeneous representation of meta-concepts}

\ivano{dire anche che col nostro approccio semplifichiamo la scrittura di trasf (prima, la complessit� dei livelli introdotta dal profiling, 
ce la portavamo dietro anche quando sviluppavamo le trasf.)}