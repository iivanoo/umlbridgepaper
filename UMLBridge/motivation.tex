\section{Motivational background}\label{sec:motivation}
\ivano{1 pagina descrizione della situazione attuale SENZA il nostro bridge, spiegare il motivo per cui ci serve il nostro bridge}
\marco{non ho inserito il caso sysml che avevamo discusso, gi� ne abbiamo 4 di scenari, se metto pure quello il paragrafo si allunga troppo}

The motivation that led us to develop a fully automatic bridging tool between UML profiles and metamodels comes from our observation and analysis of the actual state of art of the techniques used in the MDE process. We can notice a spreading production of models based on UML profiling approach and metamodeling one. What we we witnessed is a need of practical bridging tools to reconcile both worlds or to provide migration paths. A UML developer, which want to harness the model driven tool support in the design phase, may have a large legacy of UML profiled artifacts and he usually have to manually transform their profiles, generating metamodel in a domain specific language, and then generate tranformations at the modeling level to bridge his models. That's the first need we wanted to satisfy with our bridging tool. There are several more complex scenarios in which our tool may be useful. We categorize our observations in the following scenarios:

\textbf{First scenario - Synergetic approach of UML notation and MOF metamodeling}

As suggested from several studies, an approach based on the synergy of UML, as notation, and MOF-based DSLs for the engines is possible and desirable. Orlando Avila et al. proposes in its studies 

\marco{riferimento ad avila},

a separation of concerns in MDE not only in terms of multiple levels of abstraction, but also by modeling multiple concerns at any of those levels. Each of these concerns must be addressed with its own well defined and precisely focused independent Domain Specific Language. For example, the treatment of security separately from the rest of concerns. This is possible by defining a DSL specifically tailored to describe such a concern at the required level of abstraction. In other words, he proposes to generate lightweight ad-hoc editors for any aspect of the model that needs to be separately presented and edited. Our approach provides a solution in this way. By means of an automatic bridge one developper can take advantage of the MOF languages, and overcome the lack of specific tooling, which is a known disadvantage of MOF languages and the stenght of UML.

\textbf{Second scenario - large scale profile for defining application in specific constrained languages}

In Madeleine Faug�re et al, \cite{Madeleine} there is described the relation between MARTE profile and AADL-based models. AADL is a very specific and constrained language which gives a specific abstraction layer in the development life cycle of the system. MARTE is the UML profile for modeling and analysis of real-time and embedded systems. Using Marte for modelling AADL applications allows designers to model applications at earlier design stages. Every model can be completed from different point of views (like performance and scheduling, hardware and software resources modelling, etc.) making the application understanding easier. A bridging tool may help developers to define thair own application by means of MARTE and then exploit the AADL specific and constrained language.

\textbf{Third scenario - domain-specific environment from UML profiles}

Creating a MOF based domain specific modeling language, which is a formal way for describing a specific domain, may be easier using a graphical notation. As suggested in 

\marco{riferimento DOMAIN-SPECIFIC MODELING ENVIRONMENT BASED ON UML PROFILES}

, MagicDraw may be a useful tool for defining a UML profile to describe a domain. To have a DSL environment they suggest a seven step process for creating it from a profile. In our approach, one developer may just define his own profile(s) and/or model(s) and then migrating in a MOF based domain specific environment by means of an automatic transformation procedure.

\textbf{Fourth scenario - transforming among several architectural notations}

In the last years we witnessed the proliferation of architectural languages, each one with the aim of becoming the ideal language for specifying software architectures. What is evident nowadays, instead, is that architectural languages are defined by stakeholder concerns. Capturing all such concerns within a single notation is difficult and nevertheless it is impractical to define and use a \textit{universal} notation. A powerful framework to create interoperability has been proposed in \ref{duallyTSE}. This framework allows software archtiects to transform among formal ADLs and UML model-based notations and viceversa. The transformation engine has, though, to take care of the dual metamodel (UML and ECORE), this means additional operational effort while performing the transformation. Our bridge make possible to lighten the engine by working only one one metamodel (the ECORE one) thus improving the whole framework.