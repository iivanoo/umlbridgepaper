\documentclass[orivec]{llncs}

\usepackage{algorithmic}
\usepackage{pdfsync}
\usepackage{graphicx}
\usepackage{wrapfig}

\newcommand\dually{{\bf D}{\sc U}{\bf AL}{\sc L}{\sc y}}
\newcommand\name{{\sc byADL}}
\newcommand\charmy{{\sc Charmy}}

% partial maps
\newcommand{\pto}{\rightarrowtail}
% domain of a partial map
\newcommand{\pmapincluded}{\subseteq}
% typed graphs: underlying graph
\newcommand{\tgr}[1]{\ensuremath{|{#1}|}}
% typed graphs: underlying morphism
\newcommand{\tmap}[1]{\ensuremath{\tau_{{#1}}}}
% category of T-typed graphs with partial morphisms
\newcommand{\PTyped}[1]{\ensuremath{{#1}\mbox{-}\mathbf{PGraph}}}
% domain of a partial map
\newcommand{\dom}[1]{\ensuremath{\mathit{dom}(#1)}}

\def \mathrule #1#2#3{\begin{array}{l}%
    {\mbox{\scriptsize ({\sc #1})} }%
    \\ \irule{#2}{#3}%
\end{array}}
\newcommand{\irule}[2]{\frac{\textstyle\rule[-1.3ex]{0cm}{3ex}#1}%
{\textstyle\rule[-.5ex]{0cm}{3ex}#2}}

\def \mathaxiom #1#2{\begin{array}{l}%
    {\mbox{\scriptsize ({\sc #1})} }%
    \\ \iaxiom{#2}%
\end{array}}
\newcommand{\iaxiom}[1]{\textstyle\rule[-1.3ex]{0cm}{3ex}#1}

\def \mathruleside #1#2#3#4{\begin{array}{l}%
    {\mbox{\scriptsize ({\sc #1})} }%
    \\\irule{#2}{#3} \;\;\scriptstyle{#4}%
\end{array}}

% *** ALIGNMENT PACKAGES ***
%
\usepackage{array}
\usepackage{mdwmath}
\usepackage{mdwtab}
\usepackage{amsmath}
\usepackage{amssymb}
\usepackage{url}
\usepackage{times}

% Macros for proof-reading
\usepackage[normalem]{ulem} % for \sout
\usepackage{xcolor}
\newcommand{\ra}{$\rightarrow$}
\newcommand{\ugh}[1]{\textcolor{red}{\uwave{#1}}} % please rephrase
\newcommand{\ins}[1]{\textcolor{blue}{\uline{#1}}} % please insert
\newcommand{\del}[1]{\textcolor{red}{\sout{#1}}} % please delete
\newcommand{\chg}[2]{\textcolor{red}{\sout{#1}}{\ra}\textcolor{blue}{\uline{#2}}} % please change

% Put edit comments in a really ugly standout display
\usepackage{ifthen}
\usepackage{amssymb}
\newboolean{showcomments}
\setboolean{showcomments}{true} % toggle to show or hide comments
\ifthenelse{\boolean{showcomments}}
  {\newcommand{\nb}[2]{
    \fcolorbox{gray}{yellow}{\bfseries\sffamily\scriptsize#1}
    {\sf\small$\blacktriangleright$\textit{#2}$\blacktriangleleft$}
   }
   \newcommand{\version}{\emph{\scriptsize$-$working$-$}}
  }
  {\newcommand{\nb}[2]{}
   \newcommand{\version}{}
  }

\newcommand\ivano[1]{\nb{Ivano}{#1}}
\newcommand\henry[1]{\nb{Henry}{#1}}
\newcommand\marco[1]{\nb{Marco}{#1}}



% end Macros for proof-reading


\begin{document}

\title{{Automatically bridging UML profiles to MOF metamodels}
\thanks{This work is partly supported by the Italian PRIN d-ASAP project.}}

\titlerunning{Automatically bridging UML profiles to MOF metamodels}

\author{Ivano Malavolta, Henry Muccini, Marco Sebastiani}
\institute{University of L'Aquila, Dipartimento di Informatica\\
\email{\{ivano.malavolta,henry.muccini,marco.sebastiani\}@univaq.it}}

\authorrunning{Malavolta, Muccini, Sebastiani}\maketitle

\begin{abstract}
In Model Driven Engineering, UML profiles and MOF-based Domain Specific Modeling Languages are the most used approaches for describing domain specific applications.  
The choice of the right approach depends on several aspects, such as tool support, expressivity, complexity of models, company policies. In general, profiled UML models are very much used since they are intuitive for designers and model editors already exist, however they are intrinsically complex for model manipulation (e.g., transformation, analysis) is complex; conversely, DSML models are more concise and easy to be manipulated, but they require an initial effort in terms of designers training and model editors development. 
%What happens today is that UML is widely used by designers, but the underlying tools suffer from the complexity of the models. 

In this paper we propose an approach that allows to get the best of the two worlds: 
on one side designers describe the system using a UML profile familiar to them, on the other side DSML models (automatically generated from profiled UML models) enable a better model manipulation. Our approach is based on an automatic bridge between UML profiles and MOF metamodels (which are the main artifacts of MOF-based DSMLs). The bridge is transparent to the user since it autonomously operates both on UML profiles 
%(and their associated MOF metamodels) 
and all the involved models. The bridge is realized through model transformation techniques in the Eclipse platform. In this paper we show its application on a case study based on SysML.


%In Model Driven Engineering, the use of UML profiles and Domain Specific Modeling Languages are the most used approaches for specifying and documenting a system. The choice of the right approach depends on several aspects, such as tool support, expressivity, complexity of models, company or organization policies. What happens today is that in many UML-based approaches, models are intrinsically complex and the application of a profile on them exarcebates this situation; this complexity affects also the other artifacts (e.g., model transformations, graphi) associated to the
%
%
 %profiling and the use of a DSML are mutually exclusive. This precludes the possibility to benefit from both techniques while design the system of interest.
%This paper presents an approach for bridging UML profiles and MOF metamodels with a focus on automation. Indeed, the proposed mechanism is fully automatic and operates at both metamodelling and modelling levels. The approach is realized through model transformation techniques in the Eclipse platform.
\end{abstract}

%-------------------------------------------------------------------------
\section{Introduction}\label{sec:intro}
\ivano{introduzione}
\section{Motivation}\label{sec:motivation}

%UML profiling and MOF-based Domain Specific Modeling Languages are the most used approaches for specifying and documenting a system. 
%Even though these two modeling approaches share many common aspects, each of them has its own set of peculiar features. 
%Next section provides some background information on them and will serve as a dictionary of terms used
%throughout the paper. In Section \ref{sec:motivation2} we discuss the main motivations that lead us to propose this work.
%
%
%\subsection{UML profiles and Domain Specific Modeling Languages}\label{sec:background}
%\ivano{descrizione di cosa e' un profilo UML e come sono utilizzati}
%
%\ivano{descrizione di cosa e' un DSML e come sono utilizzati}

As previously said, the primary motivation for proposing our bridge is to alleviate the accidental complexity of manipulating profiled UML models, 
without forcing designers to do not use UML for their modeling activities.
In this section we provide a discussion on the main motivations that lead us to propose this work.
For the sake of clarity, we categorize our observations in three high-level scenarios. 

\textbf{First scenario - Analysis tools based on UML profiles}
Both in academia and in practice there are many verification engines that allow designers to perform some kind of analysis on UML models.
For example, the XX allows to \cite{perfMarte}, and YYY performs\cite{securityUMLsec}...

The typical workflow is that designers develop a UML model, then they augment it with additional information by means of a UML profile,
and finally the verification engine performs the analysis step on the profiled model.
These tools share a strong difficulty in manipulating models conforming to UML (vedere metaPruning).
The proposed bridge helps tool developers since they can reason on standard MOF metamodels and the involved tools
operate on smaller MOF-based models only, ...

\textbf{Second scenario - Model transformation tools}
Current model transformation tools do not natively support the management of UML profiles.
Thus, in current state-of-the-practice model transformation engines 
like ATL\footnote{ATL project website: \small{\url{http://www.eclipse.org/atl}}} and
MediniQVT\footnote{Medini QVT project website: \small{\url{http://projects.ikv.de/qvt}}}
need to be tailored or extended to support the transformation of profiled UML models; 
even worse, some interesting model transformation engine (e.g.,   
JTL\footnote{JTL project website: \small{\url{http://jtl.di.univaq.it}}}, GReAT) 
has the limitation that it cannot natively access profile-specific information when transforming UML models.
These problems are caused by a set of additional constraints that make the manipulation of profiled UML models difficult for both model transformations users and transformation engines developers:
%
\begin{itemize}
	\item specific mechanisms to apply (and un-apply) either UML profiles or stereotypes must be implemented 
	in the model transformation engine itself (or at least as an ad-hoc extension for it);
	\item the model transformation engine must consider the order in which profiles and stereotypes can be applied; 
	for example, a UML restriction imposes that a stereotype cannot be applied to a model element $x$ before 
	the profile has been applied to a package containing $x$;
	\item the model transformation language must expose specific constructs for accessing tagged values associated to a UML model element;
\end{itemize}
%
By automating the transition from profiled UML models to MOF-based models and vice versa, our bridge relaxes the above mentioned constraints
by allowing model transformation users and transformation engine developers to assume they work on MOF-based models only.
  
\textbf{Third scenario - Homogeneous representation of meta-concepts}
there are some tools that work with both UML profiles and metamodels...
\ivano{dire anche che col nostro approccio semplifichiamo la scrittura di trasf (prima, la complessit� dei livelli introdotta dal profiling, 
ce la portavamo dietro anche quando sviluppavamo le trasf.)}
\section{The automatic bridge}\label{sec:framework}
\ivano{descrizione completa del bridge senza dettagli}

The concept of bridge is used to denote the capability of interoperability between the UML approach and the DSL one. When a corresponding bridge is available for a metamodel and a corresponding profile, it is possible to produce automatically an UML profiled model from a model conforming to the metamodel in question and conversely. In our project we implement a bridge between the metamodels, either between models. The resulting artifacts of the bridge are: a metamodel which represents the profile we are about to translate,  automatically generated transformations which bridge a UML model conforming the profile in question and a model conforming to the brand new generated metamodel.

\marco{io ci metterei la figura che rappresenta schematicamente il bridge, quella che descrive quali sono gli input, qual e� il bridge ad alto livello e cosa restituisce. Se va bene la inserisco e chiudo questo paragrafo descrivendola come di seguito}
\marco{senza figura questa parte la cancello -->}

In the below figure,  source models are white, representing the input artifacts to be transformed, target models are red, to denote that they�re automatically generated. The yellow shapes denote the bridge engine. As it can be noticed, the bridging procedure is split in two phases: the first one, the yellow ellipse in figure, works at the metamodeling level by generating a metamodel which is the representation of the profile (and the UML metamodel); the second phase, the yellow arrow in figure, works at the modeling level by automatically generating transformations from a UML model which has applied a profile to a model which is conform to the brand new generated metamodel.

\marco{<--senza figura questa parte la cancello}

The resulting metamodel can be considered a union of the UML metamodel and the one represented by the profile model. We propose a further optimization to our tool which lets the developper to reduce the size of the UML metamodel imported in the target one.

\marco{metto due righe dicendo cosa si dira� nel seguito del paragrafo?}

In the following we will show the bridge behaviour at the metamodeling level and at the modeling one. In the end we will show the procedure we propose to slice the obtained metamodel.

\subsection{The bridge at the metamodeling level}\label{sec:metamodelLevel}
\ivano{livello meta con esempietto}
At the metamodeling level the bridge takes as input the UML metamodel and the profile. As shown in the previous figure 
\marco{nel primo paragrafo}
, it generates a metamodel by coping the UML metamodel in ECORE formalism, and by adding futher constructs which represent the elements in the profile. The figure 
\marco{inserire riferimento}
represents a UML profile translated into an EMF metamodel. All the UML metamodel elements are in the target metamodel, all the profiled elements (stereotypes and further data types) are represented in the target metamodel. The extension mechanism of a stereotype and its metaclass is represented by means of a binding mechanism. Every super class has a reference to the eventual stereotype.

\subsection{The bridge at the modeling level}\label{sec:modeLevel}
\ivano{livello modello con esempietto}

\section{Slicing the obtained metamodel}\label{sec:slicing}
\ivano{slicing}
\section{Implementation of the bridge}\label{sec:tool}
\ivano{dettagli sulle tecnologie usate, tipo ATL, EMF, ecc.}

\ivano{dettagli tecnici dell'implementazione}


\section{Case study}\label{sec:caseStudy}
In this section we show the application of the proposed UML bridge to a non-trivial case study based on the SysML profile.
SysML is a general-purpose modeling language for systems engineering applications \cite{sysml}, it has been proposed by the OMG group
and supports the specification of  hardware, software, processes, and facilities of a system.
The objective of this case study is to present how each aspect of the proposed approach works in practice on SysML models.

\begin{figure}[htbp]
	\centering
		\includegraphics[width=1\textwidth]{figures/caseStudy.png}
	\caption{Overview of the HSUV case study}
	\label{fig:caseStudy}
\end{figure}

As shown in Figure \ref{fig:caseStudy}, we organized the case study as a seven-steps process:
%
\begin{enumerate}
	\item transformation of the SysML profile into a MOF metamodel called $MM_{sysml}$;
	\item automatic generation of the model transformation that creates MOF-based models from SysML-profiled models;
	\item creation of an annotation model $am_{sysml}$ to slice $MM_{sysml}$;
	\item execution of \textit{MMslicer} and \textit{Tslicer} according to the $am_{sysml}$;
	\item design of a SysML-profiled UML model ($HSUV.uml$ in figure);
	\item transformation of $HSUV.uml$ into its MOF-based counterpart ($HSUV.xmi$);
	\item development of a simple manipulation tool that works on $HSUV.xmi$.
\end{enumerate}
%
Before going into the details of each step, it is important to note that steps 1-4 are executed only once, 
then the generated ATL transformation can be re-used every time a SysML model has to be bridged.

\ivano{dire che due to space limitations non mettiamo le cose che valgono in fase di generazione, ma che qui ci focalizziamo sui modelli e la trasf finale.}

In this case study we use the Papyrus modeling tool\footnote{Papyrus UML web site: \small{\url{http://www.papyrusuml.org}}} as UML
tool. We chose it for two main reasons: (i) a SysML add-in provides support for modeling SysML-like models in Papyrus according to
the official SysML specification,
and (ii) it is based on Eclipse, so our bridge and the Papyrus tool coexist in the same modeling environment.

\textbf{Step 1.} We firstly consider the SysML profile definition made available by the Papyrus add-in,
and then we transform it into the corresponding MOF metamodel
by means of the $UMLprofile2MOF$ transformation (it is described in Section \ref{sec:metamodelLevel}). 
The resulting $MM_sysml$ metamodel is composed of XX metaclasses, YY attributes and ZZ references 
(either associations, aggregations or generalizations).

\textbf{Step 2.} In this step we execute the higher-order transformation called $UML2MM_xGenerator$ in order to obtain the
model transformation that takes as input SysML-profiled models and returns MOF-based models. The generated transformation 
($UML2MM_{sysml}$ in figure) is composed of XXX transformation rules, YYYY feature bindings, ZZZ helpers; the total size of the transformation is XXXX lines of ATL code.

\textbf{Step 3.} In order to slice the obtained metamodel and transformation, 
we (automatically) generate an annotation model by means of the second generation mechanism
(see Section \ref{sec:slicing}) by assuming that only XXX and YYY diagrams are used.
The generated annotation model ($am_{sysml}$ in figure) contains XXXX links to metaclasses like YYY, XXX, and so on.

\textbf{Step 4.} The newly created annotation model is used to execute the \textit{MMslicer} and \textit{Tslicer} transformations.
These transformations adapt $MM_{sysml}$ and $UML2MM_{sysml}$ by leaving out those elements that do not belong either to XXX or YYY diagrams. 
The adapted metamodel ($slicedMM_{sysml}$) contains XXX metaclasses, yy attributes and ZZ references. 
The size of the adapted transformation ($UML2slicedMM_{sysml}$) is XXXX lines 
of ATL code; it contains XXX transformation rules, YYYYY feature bindings and ZZZ helpers. 
The $UML2slicedMM_{sysml}$ transformation will be used to transform the initial UML model described in the next step into its 
corresponding MOF-based model.

\textbf{Step 5.} In order to XXX, we decided to reuse an already existing SysML-based UML model. So we in this case study we consider
 the example model provided in the SysML official specification and we developed it in the papyrus modeling tool.
This model represents a Hybrid gas/electric powered Sport Utility Vehicle (HSUV) by focussing on its requirements, performance, structure, and behavior. Figure \ref{fig:hsuvUML} shows a fragment of ..... \ivano{completare}. Due to space limitations, we do not provide the details of the HSUV model, however the interested reader can download and use the full UML model from our UML bridge web page. 

\begin{figure}
  \centering
  \subfloat[]{\label{fig:hsuvUML}\includegraphics[scale=0.4]{figures/hsuvUML.png}}
 \hspace{10mm}
  \subfloat[]{\label{fig:hsuvMOF}\includegraphics[scale=0.4]{figures/hsuvMOF.png}}
  \caption{HSUV system: a UML model (a) and its corresponding MOF-based model (b)}
  \label{fig:hsuv}
\end{figure}
%
\textbf{Step 6.} At this point we can execute the $UML2slicedMM_{sysml}$ Transformation in order to obtain a MOF-based representation
of the HSUV model. Figure \ref{fig:hsuvMOF} shows a fragment of the .... \ivano{completare}.

\textbf{Step 7.} In this step we provide an extremely simplified manipulation tool that operates on bridge MOF models. 
It is important to point up that technical accuracy and XXX are not in the focus of this part of the case study, 
its main goal is to show as clearly as possible how manipulation tools may benefit from our UML bridge. 
More specifically, the manipulation tool is implemented as an ATL transformation that XXXX...
%
\begin{lstlisting}[breaklines,style=AMMA,language=ATL,mathescape,rulesepcolor=\color{black},caption=ATL transformation working on MOF-based SysML models,captionpos=b,label={lst:manipulationTool}]
helper context OclAny def : UntypedGenericKernels : 
  Set(GenericKernel) = 
  MM!GenericKernel.allInstancesFrom('IN')
    ->select(x | x.type.oclIsUndefined());
...
entrypoint rule KernelTypeUpdate {
  do {
    for (e in thisModule.UntypedGenericKernels()) {
      e.type<-thisModule.createNewKernelType(e);
      thisModule.setSuperTypes(e.type);
      ...
}}}
\end{lstlisting}

Listing \ref{lst:manipulationTool} shows an excerpt of this transformation. 
\ivano{il dominio della trasf e' piccolo e ben definito, accedere agli stereotipi e' piu facile, accedere ai valori dei tagged value e' piu facile}

\ivano{il case study e' riproducibile: basta andare sulla webpage e scaricarsi i modelli e trasf descritti in questa sezione}



%The objective of this case study is to present how the UML bridge works at both metamodeling and modeling
%abstraction layers, how the slicing mechanism works in practice, and how a XXX manipulation tool may benefit from
%the application of our approach.
\section{Related work}\label{sec:related}

Both UML profiling and DSML-based approaches are broadly used in practice, and the need to reconcile these two
modeling techniques is well recognized in research. Many approaches to bridge UML profiles and MOF metamodels have been proposed, each of them with its own specific features and usage scenarios. They mainly differ from our bridge in terms of level of automation, and of the effort required by designers to use them.

Abouzahra et al.~\cite{Abouzahra} proposed an integration process starting from a UML profile and a MOF metamodel.
Their concepts (e.g., stereotypes and tagged values for UML profiles, and metaclasses and structural features for MOF metamodels)
are linked by means of a weaving model. Such a model is also produced by the user and specifies a set of links between elements
of the UML profile and elements of the MOF metamodel. The tool takes as input a UML profile, the MOF metamodel, and the weaving model linking them, and it generates two ATL transformations that enables the bridge to be executed at the modeling level. Similarly to our work, the ATL transformations are generated by means of higher-order transformations. Another commonality with our work is that their approach works both at the metamodeling and modeling level, and that both are implemented in the context of the AMMA platform. In spite of these commonalities, the two approaches are very different. While in our approach a UML profile is the only input required, the approach of Abouzahra et al. requires designers to define the MOF metamodel and the weaving model linking it to the initial UML profile too.
Moreover, while in our approach the links between UML profiles and MOF metamodels are defined at the MOF level, once forever, independently from the UML profile to be bridged, in Abouzahra et al. correspondences between each UML profile and its corresponding metamodel must be manually defined; that is, there is no normalized way to define the mappings between UML profiles and MOF metamodels.

In~\cite{Wimmer}, a specular approach is proposed: the authors assume to start from the DSML metamodel, and then to automatically
generate its corresponding UML profile; model transformations for transforming DSML models to UML models and vice versa are also generated. Similarly to Abouzahra's work, this approach is based on the manual definition of a weaving model which maps each element of the DSML metamodel to the corresponding element in the UML metamodel. Again, our approach differs from this related work since the mapping between the DSML metamodel and the UML profile is defined at the MOF level, thus enabling us to propose a fully automatic mechanism. Moreover, the two approaches proposed in~\cite{Abouzahra} and~\cite{Wimmer} share the lost of information common limitation during round-trip: since the weaving model is defined by the user, elements for which there is no mapping in the weaving model are lost while transforming back and forth from UML to the DSML metamodel. Basically, our approach does not suffer from this limitation since the DSML metamodel is automatically generated and the mappings are defined at the M3 level. The only case in which our approach may experience the lost of information problem is when the slicing mechanism is used with wrong assumptions made by the designers. For example, let us suppose that a designer assumes that only UML component diagrams are of interest, and so elements pertaining to other UML diagrams are sliced out from the generated MOF metamodel. In case a state machine is defined for each component, when transforming UML to the corresponding MOF metamodel, information about state machines is lost (since sliced out). Clearly, in this scenario the lost of information problem is unavoidable, and related to the designer wrong assumption.

In~\cite{Graaf}, UML is used as a notation to visualize models conforming to a given MOF-based DSL.
More specifically, this approach presents a unidirectional mapping from models conforming to the DSL to profiled UML models.
By means of this technique it is possible to visualize DSL models into standard UML editors;
this transformation is performed for documentation purposes because designers are more accustomed to the UML concrete syntax.
This approach differs from ours in many points: (i) the whole process starts from the DSL metamodel, while the initial artifact of our bridge is the definition of the UML profile, (ii) its goal is to support the documentation of DSL models, while our bridge aims at supporting the manipulation of UML models, and (iii) it proposes a unidirectional transformation from MOF to UML, while the transformations of our bridge operate in both directions (i.e., from UML to MOF and vice versa).

In the context of the Eclipse platform an automatic mechanism for bridging UML profiles and MOF metamodels is provided by EMF. A Java class called \texttt{Profile2EPackageConverter} implements such a mechanism and  can be executed via a standard Java method call.
This class converts UML profiles to representative Ecore packages. The transformation logic is similar to the one we proposed at the metamodeling level (in Section~\ref{sec:metamodelLevel}), but it does not provide any mechanism to execute the bridge at the modeling level. This implies that designers can use \textit{Profile2EPackageConverter} to automatically convert a UML profile into an Ecore metamodel, but they are forced to consider each UML model and to manually rebuild it as a model conforming to the newly created Ecore metamodel. 
\section{Conclusions}\label{sec:conclusion}
In this paper we presented an automatic bridge between UML profiles and MOF metamodels.
The problem that we want to alleviate is due to the fact that, even if profiled UML models are very used since they are intuitive for designers and model editors already exist, 
they are intrinsically complex for model manipulation.
So, the main goal of the proposed bridge is to support those kind of projects in which the system is modelled using UML profiles, and there is a strong need of automatic model manipulation.

The bridge is fully automatic and
%, unless designers do not make particular assumptions on the models 
%(see the slicing mechanism), 
loss-less with respect to the information provided in the models. 
These two aspects make the use of the proposed bridge totally transparent for designers, allowing
model manipulation tool developers to gain from its features. The bridge is coupled with a slicing mechanism
that allows to obtain smaller and more concise target MOF metamodels.
In order to test its capabilities, we applied the proposed bridge to a non-trivial and widely used profile:
SysML. 

If on one side the application to a real-sized profile provided interesting insights and proved the usefulness of the bridge, on the other side it unveiled some of its \textbf{\textit{limitations}}.
For example, the current implementation of the bridge is able to manage nested profiles (i.e., profiles containing other profiles), but it
cannot manage profiles referencing external profiles.
Furthermore, the slicing mechanism for model transformations can be enhanced in different ways. For example, currently it deletes all the
elements (e.g., transformations rules, feature bindings, etc.) referring to meta-elements that have been sliced out from the metamodel;
this behavior is correct and straightforward to be implemented, but we also recognize that it is very basic and something more elaborated should
be provided. \ivano{other limitations?}

The primary \textbf{\textit{future work}} direction for our UML bridge is to overcome the above mentioned limitations.
Furthermore, we are also studying how to customize the generation of the target MOF metamodel by deciding a priori the number of levels in its generalizations hierarchy, by specifying whether or not some kind of attributes should be part of it or, more in general, to specify a priori some kind of characteristics that
the target MOF metamodel must have. 
We are also planning to evolve the slicing mechanism by providing a richer annotation model; for example,
designers can define a kind of black-list of meta-elements that must not be part of the sliced metamodel. 
%Another interesting research direction concerns
%the study of a mechanism for analyzing the UML "metamuddle" and then to automatically return a kind of \textit{simpleUML} metamodel with an higher-level of abstraction than UML. 
Moreover, we are planning to study our bridge in the context of model-based analysis and to check
how the proposed bridge can be combined with regression analysis techniques.
In conclusion, we are planning to embed the bridge in stable releases of other research tools; this gives us two main benefits:
(i) on one side those research tools will benefit from the features of the UML bridge, 
(ii) on the other side it also gives us the possibility to continuously test (and enhance) the UML bridge itself.




%\section*{Acknowledgments} This work is partly supported by the
%EU IST CONNECT ({\small\url{http://www.connect-forever.eu}}) No
%231167 of the FET - FP7 program and the Italian PRIN d-ASAP
%projects.

%-------------------------------------------------------------------------
%\nocite{ex1,ex2}

%\nocite{*}
\bibliographystyle{splncs}
\bibliography{umlBridge}

%\newpage
%
\section{Appendix} 

The demonstration will be carried out using two projectors to
provide both a technical and a practical perspectives in parallel.
In the following, a description of the demonstration is given steps
by step.

\subsection{\em Software Architectures and Architecture Description Languages}
We will start the demonstration by giving a short introduction to
software architectures and the existent languages and tools to
describe software architectures. A table summarizing existing
approaches will be presented with the aim of providing a snapshot of
the state of the art and of the practice in this area.
In this step ADLs are divided into two main types: first-generation and second-generation ones.
The requirements a next-generation ADL should satisfy are also provided.

\subsection{\em The role of MDE technologies into next-generation of ADLs}
We will explain the power and the potentiality of Model-driven
technologies and how they can open new perspective in software
architectures and software engineering in general.

\subsection{\em Technologies overview}
 As the audience might not be
familiar with model transformation techniques and with the AMMA
platform, their characteristics will be introduced and shown in
details. The Eugenia editor creator is presented, along with  a brief sketch on
the HUTN textual notation.

\subsection{\em The \name{} framework}
The parts that compose the \name{} framework will be conceptually
shown on projector 1 and practically shown on projector 2 (see
Figure~\ref{fig:highLevelDesign}). By means of projector 2 we will
illustrate how the various parts of the \name{} framework have been implemented
and how they relate to each other within the Eclipse platform.

\subsection{\em Putting in practice \name{}}
In this presentation step we show how the conceptual features of
\name{} are applied to an illustrative case study, explaining the
usage session of our framework both for creating a next generation ADL
and for using it to model a software system.

\subsubsection{\em Composing the notations}
In this part of the demonstration we will compose (see Section {\bf Metamodels composition:}) the Darwin ADL\footnote{J. Magee and
J. Kramer. Dynamic structure in software architectures. SIGSOFT
Softw. Eng. Notes, 21(6):3 14, 1996.}
with other metamodels in order to enrich it with additional concepts.
In this step we will show how metamodels are imported into the \name{} framework
(see the {\bf Metamodels import} section in the paper).

The first step is to consider the Darwin ADL as the staging point for the
composition. Now we enrich it with additional concerns
like fault tolerance, direct link to the development process, and so
on. We do this incrementally, by firstly adding to Darwin the
idealized fault tolerant component model\footnote{G. Ferreira, C. M.
Rubira, and R. de Lemos. Explicit Representation of Exception
Handling in the Development of Dependable Component-based Systems.
In HASE01, 2001.} that is one of the solutions extensively used to
make an architecture fault tolerant. Then we integrate Darwin
extended with fault tolerance with the development process in
Business Process Modeling Notation (BPMN)\footnote{BPMN
specification: \small{\url{http://www.bpmn.org/}}} and finally the
obtained ADL is customized by adding software connectors as
first-class elements.

% IdealComponent
\noindent \emph{DarwinFT: Extending Darwin with Fault Tolerance}. We
compose the Darwin ADL with the \textit{IdealComponent}
model\footnote{D. Di Ruscio, H. Muccini, A. Pierantonio, and P.
Pelliccione. Towards weaving software architecture models. In
MBD-MOMPES '06, 2006.} for specifying software components according
to the idealized fault tolerant component model.
Figure~\ref{fig:WM_DarwinIdealComponent} is a screenshot of the
weaving model composing the Darwin and \textit{IdealComponent}
metamodels.

% BPMN
\noindent \emph{DarwinFT+BPMN: DarwinFT \& Development process in}
\emph{BPMN}. In this scenario, we show how DarwinFT can be composed
with the BPMN metamodel. Upon doing so, software architects can
associate structural parts of the system to specific development
activities. In this step we use the Eclipse STP BPMN simplified metamodel\footnote{STP BPMN simplified metamodel: \underline{http://www.eclipse.org/bpmn}} and
Figure~\ref{fig:WM_DarwinFTBPMN} is a screenshot of the weaving model
composing the DarwinFT and BPMN metamodels.

% customization
\noindent \emph{(DarwinFT+BPMN)$_{cc}$: Darwin customization}. We
customize the DarwinFT+BPMN language by adding software connectors
as first-class elements. Components may communicate also through
connectors now and connectors have associated portals that act as
architectural roles.
Figure~\ref{fig:WM_DarwinBPMNCustomization} is a screenshot of the weaving model
customizing the DarwinFT+BPMN metamodel by adding the concept of software connector.

\subsubsection{\em Generating the model migrators}
In this step we will consider the weaving model of the first scenario
(i.e., the one composing the Darwin metamodel with the \textit{IdealComponent} UML profile)
and we will generate a model migrator from it.
More specifically, we will generate the model migrator that produces a Darwin and UML model starting from
a model conforming to the composed metamodel (the one we call DarwinFT).
Figure~\ref{fig:migratorScreenshot} shows part of the generated migrator in our Eclipse
environment.
Depending on the time availability, we will consider an example model conforming to
the composed metamodel and we will execute the generated migrator
and a generic Darwin specification extractor on that model
in order to show the audience that a ready-to-use Darwin specification is produced by the migrator.

\subsubsection{\em Generating the editors}
In this section we will present the various concrete syntaxes of the language
we are developing. We will show the three editors that the current \name{} framework
is able to provide for the generated ADL; they are the tree-based (see Figure~\ref{fig:treeEditor}),
textual (see Figure~\ref{fig:textualEditor}) and graphical one (see Figures~\ref{fig:graphicalEditor1}~\ref{fig:graphicalEditor2}), respectively.

\subsubsection{\em Using the generated language}
In this part of the demonstration we will model a software system
by means of the new ADL we have constructed (DarwinFT+BPMN$_{cc}$)
using the editors generated in the previous step.
The modeled system is called Integrated Environment for Communication on Ship
(IECS)~\cite{duallyTSE}; it is based on a specification that comes from
a project developed within Selex Communications, a company mainly
operating in the naval communication domain.

IECS is a multitier environment capable of maintaining a fail-safe
communication within a military vessel. The main functionalities of
the system are: (i) provide several communication modes; (ii) manage
and distribute operative messages; (iii) manage and configure
transmission and reception over radio channel; (iv) remote control
and monitoring of the system; (v) implement communication security
techniques.

Figures~\ref{fig:treeEditor},~\ref{fig:textualEditor},~\ref{fig:graphicalEditor1},~\ref{fig:graphicalEditor2}
give an idea on the software architecture of the IECS system.
The integration of Darwin with BPMN, allows us to model both
%(see Figure \ref{fig:editors}.c)
the software architecture of the IECS system and the adopted
development strategies. The IECS software architecture is composed
of the Equipment, Workstation, Proxy, DB and CTSM components. The
type of the latter one (CTSMType in Figure~\ref{fig:graphicalEditor1}) is described using the
idealized fault tolerant component model integrated in Darwin.
Network is a software connector; this is possible thanks to the
customization of Darwin we performed (third composition scenario).
%As can beseen in figure, Darwin and BPMN are represented via two different
%views. (DarwinFT+BPMN)$_{cc}$ enables to assign software components
%of the designed SA to developer teams. In particular, let us assume
%to have two developers teams (namely, A and B), a system engineering
%team and a testing team. The dotted lines graphically render how
%Darwin components are associated to the BPMN tasks and pools
%describing the activities assigned to each team (i.e., Proxy, DB,
%and Network are assigned to the system engineering team, Equipment
%and CTSM to the development team A, and Workstation to the
%development team B). Even if these relationships are not graphically
%rendered, they exist in the IECS model and can be accessed through
%the Properties panel of both graphical editors.

\subsection{\em Relationship with the next-generation requirements}
In this step we aim to show to the audience how the newly generated ADL
relates to the requirements a next-generation ADL should satisfy.
We will analyse each requirement independently and we will show how it relates to the generated ADL
by means of practical examples.


\subsection{\em Future work}
In this step we will present the future work we envision on the \name{} framework.
We decided to present the future work on \name{} in this demonstration in order
to discuss it with the audience about it and possibly to gain useful ideas to
enhance the framework.

The future work on \name{} comprises the following directions:
\begin{itemize}
    \item We are planning to investigate more powerful and formal ways to provide semantics.
%    In the current version of \name{}, the semantics of the ADL is defined by means
%    of relationships with $A_0$.
    Indeed, since $A_0$ is minimal, there could be some problems related
    to its expressivity (e.g., the semantics of elements that do not have a corresponding one in $A_0$ is
    not precise).
    \item
    %There is no evidence that the defined operators are enough for extending any existing ADL.
    We are investigating on the completeness of the
    set of composition operators and on how to prove that a candidate operators set is complete.
    \item The graphical editors creator is the most prototypical component of \name{},
    we will investigate on how to improve it.
%    It is not able to automatically solve every possible conflict that may arise. This
%    aspect needs further investigation.
    \item \name{} allows the creation of new ADLs by extending existing ones. We will investigate on the creation of a
    new ADL from scratch (possibly starting from the $A_0$ metamodel), and we will assess its feasibility as well.
%    \item After the execution of a migrator, a model conforming to the native ADL is generated.
%    We will work on how to provide means to identify the parts of the software architecture described using the
%    generated ADL that are affected by changes made by native ADL tools.
%    This enables software architects to understand if results of performed
%    analysis are still valid even after changes or if the analysis must be re-performed.
\end{itemize}

\subsection{\em References}
Finally, the web site of \name{} will be shown to allow the
audience to know where they can download the framework and the case
studies as well as to find more details about the tool that is under
construction and the release dates.


Depending on the time availability, some of the aforementioned items
may be shortened or deleted.

%\newpage

\section{Screen Dumps}\label{sec:dumps}
%(see next pages)
%
%\begin{figure}[h]
%    \begin{center}
%    \includegraphics[scale=.45]{figures/Darwin.png}
%    \caption{Darwin metamodel}\label{fig:Darwin}
%    \end{center}
%\end{figure}
%
%\begin{figure}[h]
%    \begin{center}
%    \includegraphics[scale=.45]{figures/IdealComponent.png}
%    \caption{Ideal Component UML profile}\label{fig:customization}
%    \end{center}
%\end{figure}
\begin{figure*}[ht]
    \begin{center}
    \includegraphics[scale=.3]{figures/WM_DarwinIdealComponent.png}
    \caption{Composing the Darwin metamodel with the \textit{IdealComponent} UML profile.}\label{fig:WM_DarwinIdealComponent}
    \end{center}
\end{figure*}
%\begin{figure*}[h]
%\begin{center}
%    \includegraphics[scale=.4]{figures/BPMN.png}
%  \caption{BPMN metamodel}\label{fig:BPMN}
%\end{center}
%\end{figure*}
%
%\begin{figure}[h]
%    \begin{center}
%%    \includegraphics[scale=.45]{figures/importWizard.png}
%    \caption{Import metamodels wizard}\label{fig:importWizard}
%    \end{center}
%\end{figure}

\begin{figure*}[ht]
    \begin{center}
    \includegraphics[scale=.3]{figures/WM_DarwinFTBPMN.png}
    \caption{Composing the DarwinFT metamodel with the BPMN metamodel.}\label{fig:WM_DarwinFTBPMN}
    \end{center}
\end{figure*}

%\begin{figure*}[h]
%\begin{center}
%    \includegraphics[scale=.4]{figures/softwareConnectorMM.png}
%  \caption{Software connector metamodel}\label{fig:softwareConnectorMM}
%\end{center}
%\end{figure*}

\begin{figure*}[ht]
    \begin{center}
    \includegraphics[scale=.3]{figures/WM_DarwinBPMNCustomization.png}
    \caption{Customization of the DarwinFT+BPMN metamodel by adding software connectors.}\label{fig:WM_DarwinBPMNCustomization}
    \end{center}
\end{figure*}

\begin{figure*}[ht]
    \begin{center}
    \includegraphics[scale=.3]{figures/migratorScreenshot.png}
    \caption{The \textit{Composed2Single} migrator generated by \name{} from the \textit{Darwin\_IdealComponent} weaving model.}\label{fig:migratorScreenshot}
    \end{center}
\end{figure*}

\begin{figure*}[ht]
    \begin{center}
    \includegraphics[scale=.3]{figures/treeEditor.png}
    \caption{The tree-based editor for the generated ADL.}\label{fig:treeEditor}
    \end{center}
\end{figure*}

\begin{figure*}[ht]
    \begin{center}
    \includegraphics[scale=.3]{figures/textualEditor.png}
    \caption{The textual editor for the generated ADL.}\label{fig:textualEditor}
    \end{center}
\end{figure*}

\begin{figure*}[ht]
    \begin{center}
    \includegraphics[scale=.3]{figures/graphicalEditor1.png}
    \caption{The structural part of the graphical editor for the generated ADL.}\label{fig:graphicalEditor1}
    \end{center}
\end{figure*}
\begin{figure*}[ht]
    \begin{center}
    \includegraphics[scale=.3]{figures/graphicalEditor2.png}
    \caption{The BPMN-related part of the graphical editor for the generated ADL.}\label{fig:graphicalEditor2}
    \end{center}
\end{figure*}


\end{document}
